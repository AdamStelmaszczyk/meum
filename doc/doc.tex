\documentclass[12pt, a4paper]{article}
\usepackage[utf8]{inputenc}
\usepackage{polski}
\usepackage{hyperref}
\usepackage{float}
\usepackage{algorithm}
\usepackage{algpseudocode}
\usepackage{geometry}
\usepackage[table]{xcolor}
\title{\textbf{Algorytm ewolucyjny z populacją rosnącą w nieskończoność}}
\author{Adam Stelmaszczyk, Michał Karpiuk}
\date{\today}
\setlength{\parindent}{0in}
\makeatletter\renewcommand{\ALG@name}{}

\begin{document}
\maketitle

\section{Zadanie}

Celem zadania jest przedstawienie koncepcji, implementacja i testowanie algorytmu ewolucyjnego z populacją rosnącą
w nieskończoność, zawierającą wszystkie punkty wygenerowane do tej pory.

\subsection{Schemat algorytmu ewolucyjnego}

\begin{algorithm}[!htb]
\label{ea}
\begin{algorithmic}[1]
\Function{algorytm\_ewolucyjny}{}
	\State $P(0) \gets \{x_1, x_2, \ldots, x_n\}$
	\State $t \gets 0$
	\While{$! stop$}
    	\For{$i = 0$ \bf{to} $i = n - 1$}
    		\If{$U(1) < p_c$}
    			\State $O(t,i) \gets mutacja(crossover(selekcja(P(t), 2)))$
    		\Else
    			\State $O(t,i) \gets mutacja(selekcja(P(t),1))$
    		\EndIf
    	\EndFor
    \State $P(t+1) \gets sukcesja(P(t),O(t))$
    \State $t \gets t+1$
  	\EndWhile
\EndFunction
\end{algorithmic}
\end{algorithm}

\nocite{*}
\bibliographystyle{plain}
\bibliography{references}
\end{document}
