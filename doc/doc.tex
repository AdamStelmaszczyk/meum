\documentclass[12pt, a4paper]{article}
\usepackage[utf8]{inputenc}
\usepackage{polski}
\usepackage{hyperref}
\usepackage{float}
\usepackage{algorithm}
\usepackage{algpseudocode}
\usepackage{amssymb}
\usepackage{geometry}
\usepackage[table]{xcolor}
\title{\textbf{Algorytm ewolucyjny z populacją rosnącą w nieskończoność}}
\author{Adam Stelmaszczyk, Michał Karpiuk}
\date{\today}
\setlength{\parindent}{0in}
\makeatletter\renewcommand{\ALG@name}{}
\renewcommand\refname{Referencje}

\begin{document}
\maketitle

\section{Zadanie}

Celem zadania jest przedstawienie koncepcji, implementacja i testowanie algorytmu ewolucyjnego z populacją rosnącą
w nieskończoność, zawierającą wszystkie punkty wygenerowane do tej pory.

\section{Koncepcja}

Na początku zostanie zaimplementowany podstawowy algorytm ewolucyjny \cite{jarabas}, dalej nazywany AE, 
działający według schematu \ref{ae}:

\begin{algorithm}[!htb]
\label{ae}
\begin{algorithmic}[1]
\Function{ae}{}
  \State $P(0) \gets \{x_1, x_2, \ldots, x_n\}$
  \State $t \gets 0$
  \While{$! stop$}
    \For{$i = 0$ \bf{to} $i = n - 1$}
      \State $O(t,i) \gets$ mutacja$($krzy{\.z}owanie$($selekcja$(P(t), 2, U)))$
    \EndFor
    \State $P(t+1) \gets$ sukcesja$(P(t),O(t))$
    \State $t \gets t+1$
  \EndWhile
\EndFunction
\end{algorithmic}
\end{algorithm}

Rozwiązania $\{x_1, x_2, \ldots, x_n\}$ reprezentowane są jako wektor liczb rzeczywistych długości $D$,
gdzie $D$ to liczba wymiarów. Szukanym optimum jest minimum funkcji celu $f(x)$.

\subsection{Mutacja}

Mutacja dodaje szum gaussowski o odchyleniu standardowym do każdej współrzędnej wejściowego rozwiązania.

\begin{algorithm}[!htb]
\begin{algorithmic}[1]
\Function{mutacja}{$x$}
  \For{$i = 0$ \bf{to} $i = D - 1$}
    \State $mutant[i] \gets x[i] + \mathcal{N}(0, 1)$
  \EndFor
\EndFunction
\end{algorithmic}
\end{algorithm}

\subsection{Krzyżowanie}

Krzyżowanie otrzymuje na wejściu dwa rozwiązania rodzicielskie $x_1,x_2$ i~zwraca jedno rozwiązanie potomne. 
Wykorzystano krzyżowanie uśredniające.

\begin{algorithm}[!htb]
\begin{algorithmic}[1]
\Function{krzy{\.z}owanie}{$x_1, x_2$}
  \For{$i = 0$ \bf{to} $i = D - 1$}
    \State $potomek[i] \gets \frac{x_1[i] + x_2[i]}{2}$
  \EndFor
\EndFunction
\end{algorithmic}
\end{algorithm}

\subsection{Sukcesja}

Sukcesja na wejściu otrzymuje dwie populacje o rozmiarze $n$: aktualną $P$ oraz populację mutantów $O$.
Wyjściem jest jedna populacja o rozmiarze $n$.

\begin{algorithm}[!htb]
\begin{algorithmic}[1]
\Function{sukcesja}{$P, O$}
  \For{$i = 0$ \bf{to} $i = n - 1$}
    \If{$f(O(t, i)) < f(P(t, i)) $}
      \State $P(t+1, i) \gets O(t, i)$
    \Else
      \State $P(t+1, i) \gets P(t, i)$
    \EndIf
  \EndFor
\EndFunction
\end{algorithmic}
\end{algorithm}

\subsection{Selekcja}

Selekcja otrzymuje na wejściu zbiór rozwiązań $X$ oraz $l$ - liczbę rozwiązań do wylosowania zgodnie z danym rozkładem $R$.

\begin{algorithm}[!htb]
\begin{algorithmic}[1]
\Function{selekcja}{$X, l, R$}
  \State $wybrane \gets \emptyset$
  \For{$i = 0$ \bf{to} $i = l - 1$}
    \State $rand = R(0, |X| - 1)$
    \State $wybrane \gets wybrane \cup X[rand]$
  \EndFor
\EndFunction
\end{algorithmic}
\end{algorithm}

\subsection{AE z populacją rosnącą w nieskończoność - NAE}

Zmodyfikowany AE będziemy nazywać NAE - nieskończony algorytm ewolucjny, tzn. AE z populacją rosnącą w nieskończoność.
NAE od AE (przedstawionego na schemacie \ref{ae}) różni się jedynie selekcją. W NAE wywołanie operatora selekcji wygląda następująco:

\begin{algorithm}[!htb]
\begin{algorithmic}[1] 
\State $O(t,i) \gets$ mutacja$($krzy{\.z}owanie$($selekcja$(P, 2, R)))$  
\end{algorithmic}
\end{algorithm}

Tzn. wybieramy 2 losowe rozwiązania spośród wszystkich populacji zgodnie z rozkładem $R$.
Wersji NAE może być nieskończenie wiele, w zależności od wybranego rozkładu. 
W tej pracy założymy, ze funkcja gęstości prawdopodobieństwa rozkładu $R$ jest postaci $at^w + b$, gdzie
$a, b, w \in \mathbb{R}$ to nieujemne parametry, zaś $t \in \mathbb{N}$ to numer populacji. Przetestujemy następujące wersje NAE:

\begin{enumerate}
 \item NAEU, rozkład jednostajny $\mathcal{U}$, $a=0, b=\frac{1}{t_{max}}$.
 \item NAEP, rozkład o pierwiastkowej funkcji gęstości prawdopodobieństwa, \\$w=\frac{1}{2}$.
 \item NAEL, rozkład o liniowej funkcji gęstości prawdopodobieństwa, $w=1$.
 \item NAEK, rozkład o kwadratowej funkcji gęstości prawdopodobieństwa, $w=2$.
\end{enumerate}


\nocite{*}
\bibliographystyle{plain}
\bibliography{references}
\end{document}
